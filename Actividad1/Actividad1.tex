\documentclass[12pt]{article}
%paquete de español
\usepackage[utf8]{inputenc}
\usepackage[spanish,mexico]{babel}
%fórmulas matemáticas
 \usepackage{amsmath}
%incluir imágenes
 \usepackage{graphicx}
 
%top matter 
\title{Actividad 1}
\author{Valenzuela Chaparro Hugo de Jesús}
\date{30 de enero de 2016}
%cuerpo del documento
\begin{document}
\maketitle

\section{Péndulo simple}
El péndulo simple es una idealización del péndulo físico, en un sistema aislado, en el cual se toman en cuenta las siguientes consideraciones:
\begin{itemize}
\item La cuerda en el que el péndulo oscila no posee masa, y siempre está tensada.
\item EL cuerpo que oscila se considera una partícula puntual.
\item El movimiento es solo en dos dimensiones, y el péndulo traza un arco en su recorrido.
\item Se considera libre de fricción o de otras fuerzas disipativas.
\item El campo gravitacional no cambia conforme el péndulo aumenta la altura.
\item El soporte del péndulo no se mueve.
\end{itemize}

\begin{center}
\includegraphics[scale=0.5]{penduloEAEAEAEA.png}
\end{center}

El movimiento del péndulo se puede representar por medio de la siguiente ecuación diferencial:

$\frac{d^2 \theta}{d t^2}+\frac{g}{\ell}\sin\theta = 0$ ...... (Eq. 1)

\noindent donde $g$ es la aceleración debida a la fuerza gravitacional, $\ell$ la longitud del péndulo y $\theta$ el desplazamiento angular respecto a la vertical.

\section{Aproximando la solución para ángulos pequeños}
Pero (Eq. 1) requiere métodos númericos para aproximar su solución, pero se puede resolver anaíticamente si tomamos sólo valores para $\theta<<1$ pues de esa manera se tiene que $\sin\theta\approx\theta$.

De esta forma obtenemos la ecuación diferencial de un oscilador armónico simple:

$\frac{d^2 \theta}{d t^2}+\frac{g}{\ell}\theta = 0$

\noindent y dando las condiciones iniciales $\theta(0)=\theta_0$ y $\frac{d\theta}{dt}(0)=0$, la solución se convierte en:

$\theta(t)=\theta_0\cos(wt)$ , $\theta<<1$ , $\omega=\sqrt{\frac{g}{\ell}}$

El movimiento es armónico simple, donde $\theta_0$ es la semi-amplitud, es decir, el ángulo entre la cuerda del péndulo y la vertical. 
El periodo de la oscilación, lo que tarda en completar una ida y venida es:

$T_0=\frac{2\pi}{\omega}=2\pi\sqrt\frac{\ell}{g}$ ...... (Eq. 2)  , $\theta<<1$

\noindent usando la aproximación de los ángulos pequeños, el periodo es independiente de la amplitud $\theta_0$, y de la masa, esta es la propiedad de isocronismo que Galileo notó.

\subsection{Aproximación empírica para la longitud de péndulo}
Si despejamos $\ell$ de (Eq. 2) para el periodo, se obtiene:

$\ell=\frac{g}{\pi^2}\frac{T_0^2}{4}$

\noindent si se usa el sistema mks y si la medición se está haciendo en la superficie terrestre, por lo que $g\approx9.81\frac{m}{s^2}$ y $\frac{g}{\pi^2}\approx1$

Entonces obtenemos estas aproximaciones rasonables para el periodo y la longitud:

$\ell\approx\frac{T_0^2}{4}$

$T_0\approx2\sqrt{\ell}$

\section{Periodo de amplitud arbitraria}
Para amplitudes más grandes que las usadas en la aproximación para ángulos pequeños, podemos calcular el periodo exacto invirtiendo la siguiente ecuación para la velocidad angular:

$\frac{d\theta}{dt}=\sqrt{\frac{2g}{\ell}\cos\theta-\cos\theta_0}$

\noindent obteniendo entonces:

$\frac{dt}{d\theta}=\sqrt{\frac{\ell}{2g}}\frac{1}{\sqrt{\cos\theta-\cos\theta_0}}$

\noindent 
e integrando 4 veces sobre el cuarto de un ciclo lleva a:

$T=4\sqrt{\frac{\ell}{2g}}\int_{0}^{\theta_0}\frac{1}{\sqrt{\cos\theta-\cos\theta_0}}d\theta$

Esta integral diverge conforme $\theta_0$ se acerca a la vertical

$\lim_{\theta_0 \to \pi}T=\infty$

\noindent así que un péndulo con la energía necesaria para ir verticalmente nunca llegará. La integral se puede escribir en terminos de integrales elípticas conmo:

$T=4\sqrt{\frac{\ell}{g}}K(\sin^2{\frac{\theta_0}{2}})$........(Eq. 3)

\noindent donde K es la integral elíptica incompleta de primer tipo definida por:

$K(k)=F(\frac{\pi}{2}, k)=\int_{0}^{\frac{\pi}{2}}\frac{1}{\sqrt{1-k^2\sin^2{u}}}du$

\noindent Para comparar la aproximación con la solución lineal, consideramos el periodo de un péndulo de longitud 1 m en la Tierra ($g=9.80665\frac{m}{s^2}$) con un ángulo inicial de $10^\circ$. La diferencia entre ambos valores es menor del 0.2\%. que es mucho menor que la que podría causar de g debido a la localización geográfica.

Hay distintas maneras de procedera calcular la integral elíptica:

\subsection{Solución polinomial de Legendre}

Dada (Eq. 3) y la solución polinomial de Legendre para la interal elíptica:

$K(k)=\frac{\pi}{2}\{1+(\frac{1}{2})k^2+(\frac{1\cdot3}{2\cdot4})^2k^2+...+[\frac{(2n-1)!!}{(2n)!!}]^2k^{2n}+...\}$, 

\noindent una solución exacta para el periodo del péndulo es:

$T=2\pi\sqrt{\frac{\ell}{g}}\cdot\sum_{n=0}^{\infty}\left[\left(\frac{(2n)!}{(2^{n}\cdot{n!})^2}\right)^2\cdot\sin^{2n}(\frac{\theta_0}{2})\right]$

La siguiente figura muestra los errores relativos usando series de potencia. $T_0$ es la aproximación lineal, y de $T_2$ a $T_{10}$ incluyen los términos de la segunda a la décima potencia, respectivamente.

\begin{center}
\includegraphics[scale=0.5]{errorrelativo.png}
\end{center}

\subsection{Solución de serie de potencia}
Otra solución se puede encontrar en las series de Maclaurin:

$\sin{\frac{\theta_0}{2}}=\frac{1}{2}\theta_0-\frac{1}{48}\theta_0^3+\frac{1}{3840}\theta_0^5-\frac{1}{645120}\theta_0^7+...$

que se usa en la solución polinimial de Legendre. La serie de potencia que resulta entonces es:

$T=2\pi\sqrt{\frac{\ell}{g}}\left(1+\frac{1}{16}\theta_0^2+\frac{11}{3072}\theta_0^4+\frac{173}{737280}\theta_0^6+\frac{22931}{1321205760}\theta_0^8+\frac{1319183}{951268147200}\theta_0^{10}+\frac{233526463}{2009078326886400}\theta_0^{12}+...\right)$

\subsection{Solución de la media aritmético-geométrica}
Dada (Eq. 3) y la solución de la media aritmético-geométrica para la integral elíptica:

$K(k)=\frac{\frac{\pi}{2}}{M(1-k,1+k)}$, 

\noindent donde M(x,y) es la media aritmético-geométrica para x e y.

Esto lleva a una fórmula alternativa, rápida y convergente para el periodo:

$T=\frac{2\pi}{M(1, \cos{\frac{\theta_0}{2}})}\sqrt{\frac{\ell}{g}}$

\section{Referencias}
\begin{enumerate}
\item Robert N, M.G. Olsson. "The pendulum: Rich physics from a simple system". (1986).
\item Carvalhaes C, Suppes P. "Approximations for the period of the simple pendulum based on the arithmetic-geometric mean". (2008).
\item Adlaj S. "An eloquent formula for the perimeter of an ellipse". (2012).
\item Van Baak T. "A New and Wonderful Pendulum Period Equation". (2013).
\item Wikipedia. LINK: www.wikipedia.org
\end{enumerate}

\end{document}
